\documentclass[11pt]{article}
\usepackage[utf8]{inputenc}
\usepackage[slovene]{babel}

\usepackage{amsthm}
\usepackage{amsmath, amssymb, amsfonts}
\usepackage{relsize}
\usepackage{listings}
\lstset{
	basicstyle=\ttfamily,
	mathescape
}

\DeclareMathOperator{\dist}{dist}
\newcommand{\R}{\mathbb{R}}
\newcommand{\N}{\mathbb{N}}
\newcommand{\p}{\mathbb{P}}
\newcommand{\A}{\mathcal{A}}
\newcommand{\B}{\mathcal{B}}
\newcommand{\C}{\mathcal{C}}
\newcommand{\s}{\mathcal{S}}
\newcommand{\rr}{\mathcal{R}}
\newcommand{\F}{\mathcal{F}}
\newcommand{\OO}{\mathcal{O}}
\newcommand{\T}{\mathcal{T}}

\theoremstyle{definition}
\newtheorem{definicija}{Definicija}[section]

\theoremstyle{definition}
\newtheorem{problem}{Problem}[section]

\newtheorem{lema}{Lema}[section]
\newtheorem*{pravilo}{Pravilo}
\newtheorem{trditev}{Trditev}[section]
\newtheorem{izrek}{Izrek}[section]
\newtheorem*{algoritem}{Algoritem}
\newtheorem*{posledica}{Posledica}
\newtheorem*{opomba}{Opomba}
\newtheorem*{metoda}{Metoda}
\newtheorem*{ideja}{Ideja}

\title{Numerične metode 2 - definicije, trditve in izreki}
\author{Oskar Vavtar \\
po predavanjih profesorice Marjetke Knez}
\date{2020/21}

\begin{document}
\maketitle
\pagebreak
\tableofcontents
\pagebreak

% #################################################################################################

\section{Teorija aproksimacije}
\vspace{0.5cm}

\begin{definicija}[Aproksimacijska shema]

Z $X$ označimo (realni) vektorski prostor, katerega elemente želimo aproksimirati, $S \subseteq X$ označuje podprostor oz. podmnožico, v kateri iščemo aproksimant. \textit{Aproksimacijska shema} je operator $\A: X \rightarrow S$, ki vsakemu elementu $f \in X$ priredi aproksimacijski element $\tilde{f} = \A f \in S$.

\end{definicija}
\vspace{0.5cm}

\begin{definicija}[Optimalni aproksimacijski problem]

Naj bo $X$ vektorski prostor z normo $\|\cdot\|$, $S \subseteq X$. Za $f \in X$ iščemo $\tilde{f} \in S$, da je 
$$\|f - \tilde{f}\| ~=~ \inf_{s \in S}{\|f - s\|} ~=:~ \dist(f, S).$$

\end{definicija}
\vspace{0.5cm}

\begin{definicija}

Recimo, da je $S = S_n$, kjer je $n$ dimenzija. Zanima nas, ali za $f \in X$ in $\tilde{f_n} \in S_n$ napaka $\|f - \tilde{f_n}\|$ konvergira proti $0$, ko gre $n \rightarrow \infty$. Če je to res, je aproksimacijska shema \textit{konvergentna}.

Če gledamo zaporedje podprostorov $S_n \subset X$, mora veljati, da z večanjem svobodnih parametrov postane $S_n$ gost v $X$. Za polinome to sledi iz Weierstrassovega izreka.

\end{definicija}
\vspace{0.5cm}

\begin{izrek}[Weierstrassov izrek]

Naj bo $f \in \C([a, b])$ poljubna funkcija. Potem $\forall \varepsilon > 0$ obstaja polinom $p$, da je 
$$\|f - p\|_{\infty,~[a, b]} ~<~ \varepsilon.$$
Drugače povedano:
$$\dist_{\infty}(f, \p_n) ~\xrightarrow{n \rightarrow \infty}~ 0.$$

\end{izrek}
\vspace{0.5cm}

\begin{definicija}[Bernsteinov polinom]

$$\B_n f(x) ~=~ \sum_{i=0}^n f\left(\frac{i}{n}\right) \cdot B_i^n(x),$$
kjer je $B_i^n$ \textit{Bernsteinov bazni polinom}:
$$B_i^n(x) ~:=~ \binom{n}{i} x^i (1-x)^{n-1}, ~~~i = 0, 1, \ldots, n$$
Da se pokazati, da gre $\|f - \B_n f\|_{\infty,~[0, 1]} \xrightarrow{n \rightarrow \infty} 0$. Bernsteinov aproksimacijski polinom nam poda en možen način aproksimacije funkcije $f$ (na $[0,1]$).

\end{definicija}
\vspace{0.5cm}

\begin{definicija}[Bernsteinov aproksimacijski operator]
~\\
$\B_n: \C([a, b]) \rightarrow \p_n$, ~$f \mapsto \B_n f$:
$$\B_n f(x) ~=~ \sum_{i=0}^n f \left( a + \frac{i}{n} (b-a) \right) \cdot B_i^n\left(\frac{x-a}{b-a}\right)$$
$$\|f - \B_n f\|_{\infty, ~[a, b]} ~=~ \max_{x \in [a, b]} |f(x) - \B_n f(x)|$$ 

\end{definicija}
\vspace{0.5cm}

\pagebreak

% #################################################################################################

\section{Enakomerna aproksimacija \\zveznih funkcij s polinomi}
\vspace{0.5cm}

\begin{problem}

Za dano $f \in \C([a, b])$ iščemo polinom $p^* \in \p_n$, za katerega velja
$$\|f - p^*\|_{\infty,~[a, b]} ~=~ \min_{p \in \p_n} \|f - p\|_{\infty,~[a, b]} ~=~ \min_{p \in \p_n} \max_{x \in [a, b]} |f(x) - p(x)|.$$
Ta problem sodi pod optimalne aproksimacijske probleme. Polinom $p^*$ imenujemo \textit{polinom najboljše enakomerne aproksimacije}\footnote{Od tu naprej \textit{p.n.e.a.}} za $f$ na $[a, b]$. Problem je nelinearen.

\end{problem}
\vspace{0.5cm}

\begin{izrek}

Naj bo $f \in \C([a, b])$. Če je polinom $p \in \p_n$ takšen, da \textit{residual} $r = f - p$ doseže svojo normo $\|r\|_{\infty,~[a, b]}$ alternirajoče v vsaj $n+2$ točkah $x_i \in [a, b]$, $a \leq x_0 < x_1 < \ldots < x_{n+1} \leq b$, potem je $p$ \textit{p.n.e.a.} za $f$ na $[a, b]$. \\

\textbf{Natančneje:} Če obstaja $n+2$ točk $x_i \in [a, b]$, da je $\|r\|_{\infty, [a, b]} = |r(x_i)|$ za $i = 0, 1, \ldots, n+1$, in $r(x_i) \cdot r(x_{i+1}) < 0$ za $i = 0, 1, \ldots, n$, potem je $p \in \p_n$ \textit{p.n.e.a.} za $f$ na $[a, b]$.

\end{izrek}
\vspace{0.5cm}

\begin{definicija}

Naj bo $E = \{x_i;~a \leq x_0 < x_1 < \ldots < x_{n+1} \leq b\}$. Definirajmo \textit{minimaks} za $f$ na $E$:
$$M_n(f, E) ~=~ \min_{p \in \p_n} \max_{x \in E} |f(x) - p(x)|$$
Polinom, pri katerem je ta minimum dosežen, imenujemo \textit{p.n.e.a.} za $f$ na množici $E$. Dobimo ga tako, da rešimo sistem linearnih enačb:
$$f(x_i) - p(x_i) ~=~ (-1)^i m, ~~~i = 0, 1, \ldots, n+1$$
kjer so neznanke koeficienti polinoma $p$ ter število $m$.\footnote{Imamo $n+2$ enačb za $n+2$ neznank.}

\pagebreak

\begin{algoritem}[Remesov postopek]

Vhodni podatki: funkcija $f$, interval $[a, b]$, stopnja $n$, toleranca $\varepsilon$ \\

Ponavljaj $k = 0, 1, 2, \ldots$
\begin{enumerate}
	\item Poišči polinom ${p_k}^* \in \p_n$, ki zadošča pogojem
	$$f(x_i) - {p_k}^*(x_i) ~=~ (-1)^i m, ~~~ i = 0, 1, \ldots, n+1$$
	
	\item Poišči ekstrem residuala $r_k = f - {p_k}^*$, torej poišči $u \in [a, b]$, da bo
	$$|r_k(u)| ~=~ \|r_k\|_{\infty,~[a, b]}$$
	
	\item Če je $|r_k(u)| - |m| < \varepsilon$, potem končaj in vrni $p^* = {p_k}^*$. 
\end{enumerate}
\end{algoritem}
\vspace{0.5cm}

\begin{opomba}

Da se dokazati, da zaporedje polinomov, ki ga tvori Remesov postopek konvergira proti \textit{p.n.e.a.} $p^*$. Hitrost konvergence je \textit{linearna}.

\end{opomba}
\vspace{0.5cm}

\end{definicija}
\vspace{0.5cm}

\pagebreak

% #################################################################################################

\section{Aproksimacija po metodi najmanjših kvadratov}
\vspace{0.5cm}

\begin{definicija}

Naj bo $X$ vektorski prostor nad $\R$ s skalarnim produktom $\langle \cdot, \cdot \rangle$, s kvadratno normo $\|\cdot\|_2 = \sqrt{\langle \cdot, \cdot \rangle}$. $S \subseteq X$ je končnodimenzionalen podprostor v $X$, definiran kot
$$S ~=~ \mathcal{L}\textit{in}\{\varphi_1, \varphi_2, \ldots, \varphi_n\}, ~~~\dim{S} = n.$$
Za izbran $f \in X$ iščemo $f^* \in S$, da velja
$$\|f - f^*\|_2 ~=~ \min_{s \in S}\|f - s\|_2.$$
$f^*$ imenujemo \textit{element najboljše aproksimacije po metodi najmanjših kvadratov.}\footnote{\textit{e.n.a. po MNK}}

\end{definicija}
\vspace{0.5cm}

\begin{izrek}

Naj bo $S \subseteq X.$ Element $f^* \in S$ je element najboljše aproksimacije po MNK za $f \in X$ natanko tedaj, ko je $f - f^* \perp S$.

\end{izrek}
\vspace{0.5cm}

\begin{posledica}

Iz izreka sledi konstrukcija. Naj bodo $\varphi_1, \varphi_2, \ldots, \varphi_n$ baza podprostora $S$. 
$$f^* ~=~ \sum_{j=1}^n \alpha_j \varphi_j,$$ 
$(\alpha_j)_{j=1}^n$ so neznani koeficienti. Veljati mora $f - f^* \perp S$, torej $f - f^* \perp \varphi_i$ $\forall i = 1, 2, \ldots, n$. Na podlagi tega dobimo
$$\begin{bmatrix}
\langle \varphi_1, \varphi_1 \rangle & \cdots & \langle \varphi_n, \varphi_1 \rangle \\
\vdots & \ddots & \vdots \\
\langle \varphi_1, \varphi_n \rangle & \cdots & \langle \varphi_n, \varphi_n \rangle
\end{bmatrix} \begin{bmatrix}
\alpha_1 \\
\vdots \\
\alpha_n
\end{bmatrix} ~=~ \begin{bmatrix}
\langle f, \varphi_1 \rangle \\
\vdots \\
\langle f, \varphi_n \rangle
\end{bmatrix}.$$
Ta sistem enačb imenujemo \textit{Gramov} oz. \textit{normalni sistem}. Numerično ta sistem rešimo z razcepom Choleskega. Levo matriko imenujemo \textit{Gramova matrika}. Gramova matrika $G = (\langle \varphi_j, \varphi_i \rangle)_{i,j=1}^n$ je \textit{simetrična pozitivno definitna} matrika.

\end{posledica}
\vspace{0.5cm}

\pagebreak

\begin{algoritem}[Gram-Schmidt]

Vhodni podatki: baza $\{\psi_1, \psi_2, \ldots, \psi_n\}$. \\
Izhod: Ortonormirana baza $\{\varphi_1, \varphi_2, \ldots, \varphi_n\}$.
\begin{lstlisting}
for i = 1:n
	$\varphi_\texttt{i}$ = $\psi_\texttt{i}$
end
for i = 1:n
	$\varphi_\textit{i}$ = $\frac{\varphi_\texttt{i}}{\|\varphi_\texttt{i}\|_{\texttt{2}}}$
	for j = i+1:n
		$\varphi_\texttt{j}$ = $\varphi_\texttt{j}$ - $\langle \varphi_\texttt{j}$, $\varphi_\texttt{i} \rangle \varphi_\texttt{i}$
	end
end
\end{lstlisting}

\end{algoritem}
\vspace{0.5cm}

\pagebreak

% #################################################################################################

\section{Interpolacija}
\vspace{0.5cm}

\begin{problem}

Podane so vrednosti izbrane funkcije $f$ v $n+1$ paroma različnih točkah $x_0, x_1, \ldots, x_n$ na realni osi\footnote{To so \textit{interpolacijske točke}.}, iščemo pa neko preprostejšo funkcijo $g$, ki zadošča pogojem
$$f(x_i) ~=~ g(x_i), ~~~i = 0, 1, \ldots, n.$$

\end{problem}
\vspace{0.5cm}

\begin{problem}[Polinomska interpolacija]

Imejmo funkcijo $f \in \C([a, b])$ in zaporedje točk \\$a \leq x_0 < x_1 < \ldots < x_n \leq b$. Iščemo polinom $p = a_0 + a_1 x + \ldots + a_n x^n \in \p_n$, ki zadošča pogojem
$$p(x_i) ~=~ f(x_i), ~~~i = 0, 1, \ldots, n.$$

\end{problem}
\vspace{0.5cm}

% *************************************************************************************************

\subsection{Lagrangeeva oblika zapisa interpolacijskega polinoma}
\vspace{0.5cm}

\begin{definicija}[Lagrangeevi bazni polinomi]

\begin{align*}
\ell_{0, n}(x) ~&=~ \frac{(x-x_1)(x-x_2)\ldots(x-x_n)}{(x_0-x_1)(x_0-x_2)\ldots(x_0-x_n)}\\
\ell_{1, n}(x) ~&=~ \frac{(x-x_0)(x-x_2)\ldots(x-x_n)}{(x_1-x_0)(x_1-x_2)\ldots(x_1-x_n)}\\
&~~\vdots \\
\ell_{n, n}(x) ~&=~ \frac{(x-x_0)(x-x_1)\ldots(x-x_{n-1})}{(x_n-x_0)(x_n-x_1)\ldots(x_n-x_{n-1})}
\end{align*}
$i$-ti Lagrangeev bazni polinom lahko posplošimo kot
$$\ell_{i, n}(x) ~=~ \prod_{\substack{j=0 \\ j \neq i}}^n \frac{x-x_j}{x_i-x_j}, ~~~i = 0, 1, \ldots, n$$
Velja:
$$\ell_{i, n}(x_j) ~=~ \begin{cases}
1; ~&i=j \\
0; ~&i \neq j
\end{cases}$$

\end{definicija}
\vspace{0.5cm}

\begin{lema}

Polinomi $\ell_{i, n}$, $i = 0, 1, \ldots, n$ so baza za $\p_n$.

\end{lema}
\vspace{0.5cm}

\begin{trditev}[Lagrangeeva oblika zapisa interpolacijskega polinoma]

$$p(x) ~=~ \sum_{i=0}^n f(x_i) \ell_{i, n}(x)$$

\end{trditev}
\vspace{0.5cm}

\begin{lema}

Če je $f \in \p_n$, potem je
$$\sum_{i=0}^n f(x_i) \ell_{i, n}(x) ~=~ f(x).$$

\end{lema}
\vspace{0.5cm}

\begin{posledica}

Lagrangeevi bazni polinomi tvorijo \textit{particija} oz. \textit{razčlenitev} enote:
$$\sum_{i=0}^n \ell_{i,n}(x) ~=~ 1.$$

\end{posledica}
\vspace{0.5cm}

\begin{izrek}

Naj bo $a \leq x_0 < x_1 < \ldots < x_n \leq b$, $f \in \C^{n+1}([a, b])$ in \\$p(x) = \mathlarger{\sum_{i=0}^n f(x_i) \ell_{i, n}(x)}$ interpolacijski polinom za $f$ na točkah $x_0, x_1, \ldots, x_n$. Potem $\forall x \in [a, b]$ obstaka nek $\xi_x \in (a, b)$, da velja
\begin{align*}
f(x) - p(x) ~&=~ \frac{f^{(n+1)}(\xi_x)}{(n+1)!} \omega(x), \\
\omega(x) ~&=~ (x-x_0)(x-x_1)\ldots(x-x_n), ~~~\omega \in \p_{n+1}.
\end{align*}

\end{izrek}
\vspace{0.5cm}

\begin{opomba}

Za poljuben $x \in  [a, b]$ torej velja
$$\|f(x) - p(x)\|_{\infty,~[a, b]} ~\leq~ \frac{1}{(n+1)!} \|\omega(x)\|_{\infty,~[a, b]} \cdot \|f^{(n+1)}(x)\|_{\infty,~[a, b]}$$

\end{opomba}
\vspace{0.5cm}

% *************************************************************************************************

\subsection{Newtonova oblika zapisa interpolacijskega polinoma}
\vspace{0.5cm}

\begin{problem}

Za bazo v kateri bomo interpolacijski polinom izrazili, izberemo \textit{prestavljene potence}:
$$1, ~x-x_0, ~(x-x_0)(x-x_1), ~(x-x_0)(x-x_1)(x-x_2), ~\ldots, ~(x-x_0)(x-x_1)\ldots(x-x_{n-1})$$
$$p(x) ~=~ \sum_{i=0}^n c_i (x-x_0)(x-x_1)\ldots(x-x_{i-1})$$
Iščemo $c_i$, $i = 0, 1, \ldots, n$, da bo $p(x_i) = f(x_i)$ $\forall i$.

\end{problem}
\vspace{0.5cm}

\begin{definicija}

\textit{Deljiva diferenca} $[x_0, x_1, \ldots, x_k]f$ je \textit{vodilni koeficient} interpolacijskega polinoma stopnje $k$\footnote{Koeficient pri potenci $x^k$.}, ki se s funkcijo $f$ ujema v točkah $x_0, x_1, \ldots, x_k$. Sledi
$$p_l(x) ~=~ p_{k-1}(x) ~+~ [x_0, x_1, \ldots, x_k]f \,(x-x_0)(x-x_1)\ldots(x-x_{k-1}).$$

\end{definicija}
\vspace{0.5cm}

\begin{trditev}[Newtonova oblika zapisa interpolacijskega polinoma]

$$p(x) ~=~ \sum_{i=0}^n [x_0, x_1, \ldots, x_i]f \,(x-x_0)(x-x_1)\ldots(x-x_{i-1})$$

\end{trditev}
\vspace{0.5cm}

\begin{izrek}[Rekurzivna formula za deljene diference]

Naj bodo $x_0, x_1, \ldots, x_k$ paroma različne točke na $x$-osi. Tedaj je
$$[x_0, x_1, \ldots, x_k]f ~=~ \frac{[x_1, x_2, \ldots, x_k]f - [x_0, x_1, \ldots, x_{k-1}]f}{x_k - x_0}.$$

\end{izrek}
\vspace{0.5cm}

\pagebreak
\begin{algoritem}[Hornerjev algoritem]

Vhodni podatki: 
\begin{itemize}
	\item $x_0, x_1, \ldots, x_n$
	\item $d_0, x_1, \ldots, d_n$
	\item $x$
\end{itemize}
\begin{lstlisting}
v$_\texttt{n}$ = d$_\texttt{n}$
for i = n-1$\,$:$\,$-1$\,$:$\,$0
    v$_\texttt{i}$ = d$_\texttt{i}$ + (x-x$_\texttt{i}$)$\cdot$v$_{i+1}$
end
\end{lstlisting}
Izhod: $v_0$

\end{algoritem}
\vspace{0.5cm}

\begin{definicija}

Pravimo, da se polinom $p$ z $f$ ujema v točki $x_i$ \hbox{$(k+1)$-kratno}, če se ujema v vrednosti in v prvih $k$ odvodih:
$$p(x_i) ~=~ f(x_i), ~~~p'(x_i) ~=~ f'(x_i), ~~~\ldots~~~ p^{(k)}(x_i) ~=~ f^{(k)}(x_i).$$
Tak polinom $p$ stopnje $k$ je Taylorjev polinom:
$$p(x) ~=~ f(x_i) + f'(x_i)\,(x-x_i) + \frac{f''(x_i)}{2}(x-x_i)^2 + \ldots + \frac{f^{(k)}(x_i)}{k!}(x-x_i)^k.$$
Posplošitev rekurzivne formule: recimo\footnote{Vrstni red točk v diferenci ni pomemben.}, da je $x_i \leq x_{i+1} \leq \ldots \leq x_{i+k}$:
$$[x_i, x_{i+1}, \ldots, x_{i+k}]f ~=~ \begin{cases}
\frac{f^{(k)}(x_i)}{k!}; ~&x_i = x_{i+1} = \ldots = x_{i+k} \\
\frac{[x_{i+1}, \ldots, x_{i+k}]f - [x_i, \ldots, x_{i}]}{x_{i+k} - x_i}, ~&x_i \neq x_{i+k}
\end{cases}$$

\end{definicija}
\vspace{0.5cm}

% *************************************************************************************************

\pagebreak

% #################################################################################################

\section{Numerično odvajanje}
\vspace{0.5cm}

\begin{problem}

Iščemo približek za vrednost odvoda funkcije $f$ pri nekem izbranem $x$. Približek bi radi izrazili s kombinacijo vrednosti funkcije $f$ v bližnjih točkah $x_0, x_1, \ldots, x_n$.

\end{problem}
\vspace{0.5cm}

\begin{metoda}[Ideja za izpeljavo aproksimacijskih formul]

Kot približek za \hbox{odvod} funkcije $f$ v izbranem $x$ vzamemo vrednost odvoda interpolacijskega \hbox{polinoma} (ki se z $f$ ujema v točkah $x_0, x_1, \ldots, x_n$) pri izbranem $x$. Vemo že:
$$f(x) ~=~ \underbrace{\sum_{i=0}^n f(x_i) \ell_{i, n}(x)}_{p(x)} ~+~ \omega(x)[x_0, x_1, \ldots, x_n, x],f$$
kjer je
$$\omega(x) ~=~ \prod_{i=0}^n (x-x_i).$$
Odvajamo to formulo:
$$f'(x) ~=~ \underbrace{\sum_{i=0}^n f(x_i) (\ell_{i, n}(x))'}_{\text{aproksimacijska formula}} ~+~ \underbrace{(\omega(x) [x_0, x_1, \ldots, x_n, x] f')'}_{\text{napaka:}~R(f)~\text{ali}~\rr f}$$ 

\end{metoda}
\vspace{0.5cm}

\begin{opomba}[Odvod deljene diference]

$$\frac{d}{dx}[x_0, x_1, \ldots, x_n, x]f ~=~ [x_0, x_1, \ldots, x_n, x, x]f$$

\end{opomba}
\vspace{0.5cm}

\begin{trditev}[?]
\begin{align*}
f'(x_k) ~&=~ p'(x_k) ~+~ \omega'(x_k)[x_0, x_1, \ldots, x_n, x_k]f ~+~ \underbrace{\omega(x_k)}_{0} [x_0, x_1, \ldots, x_n, x_k, x_k]f \\
f'(x_k) ~&=~ p'(x_k) ~+~ \omega'(x_k) \frac{f^{(n+1)}(\xi)}{(n+1)!}, ~~~\xi \in (x_0, x_n)
\end{align*}
\end{trditev}
\vspace{0.5cm}

\pagebreak

% #################################################################################################

\section{Numerično integriranje}
\vspace{0.5cm}

\begin{problem}

Radi bi izračunal približek za integral
$$\s f ~=~ \int_a^b f(x)\,dx.$$
Približek bi radi izrazili s kombinacijo vrednosti funkcije $f$ v izbranih točkah iz intervala $[a, b]$.
\begin{align*}
\s: \C([a, b]) ~&\rightarrow~ \R \\
f ~&\mapsto~ \int_a^b f(x)\,dx
\end{align*}
Funkcional $\s$ je \textit{linearen}:
$$\s (\alpha f + \beta g) ~=~ \alpha \s f + \beta \s g, ~~~\alpha, \beta \in \R, ~f, g \in \C([a, b]).$$

\end{problem}
\vspace{0.5cm}

\begin{metoda}[Ideja za izpeljavo formul]

Namesto funkcije $f$ integriramo interpolacijski polinom za $f$ na točkah $x_0, x_1, \ldots, x_n$ iz intervala $[a, b]$, \\$a \leq x_0 < x_1 < \ldots < x_n \leq b$. Vemo:
\begin{align*}
f(x) ~&=~ p(x) ~+~ \omega(x) [x_0, x_1, \ldots, x_n]f \\
p(x) ~&=~ \sum_{i=0}^n f(x_i) \ell_{i, n}(x) \\ 
\ell_{i, n}(x) ~&=~ \prod_{\substack{j=0\\ j \neq i}}^n \frac{x-x_j}{x_i-x_j}
\end{align*}
Integriramo prvo enačbo in dobimo
$$\underbrace{\int_a^b f(x)\,dx}_{\s f} ~= \underbrace{\int_a^b p(x)\,dx}_{\text{približek za integral}~\F f} +~ \underbrace{\int_a^b \omega(x) [x_0, x_1, \ldots, x_n, x]f\,dx}_{\text{napaka}~\rr f}$$

\end{metoda}
\vspace{0.5cm}

\begin{trditev}[Kvadraturna formula oz. integracijsko pravilo]

\begin{align*}
\s f ~&=~ \F f + \rr f \\
\F f ~&=~ \int_a^b \sum_{i=0}^n f(x_i) \ell_{i, n}(x)\,dx ~=~ \underbrace{\sum_{i=0}^n f(x_i) \int_a^b \ell_{i, n}(x)\,dx}_{A_i} \\
\F f ~&=~ \sum_{i=0}^n A_i f(x_i)
\end{align*}

\begin{itemize}
	\item $A_i$ \ldots uteži integracijskega pravila
	\item $x_0, x_1, \ldots, x_n$ \ldots vozli integracijskega pravila
\end{itemize}

\end{trditev}
\vspace{0.5cm}

\begin{definicija}

\textit{Red} oziroma \textit{stopnja} integracijskega pravila je enaka $m$, če je pravilo točno za vse polinome stopnje $\leq$ $m$, to je
$$\rr p ~=~ 0 ~~~\forall p \in \p_m ~~~\text{in}~~~ \rr x^{m+1} ~\neq~ 0.$$
Glede na izbiro vozlov ločimo:
\begin{itemize}

	\item \textit{Newton-Cotesova} pravila: \\vozle izberemo ekvidistantno:
	\begin{align*}
	h ~&=~ \frac{b-a}{n}, ~~~x_0 ~=~ a \\
	x_i ~&=~ x_0 + ih, ~~~i ~=~ 0, 1, 2 \ldots, n
	\end{align*}	
	Ločimo:
	\begin{itemize}
		\item Pravila odprtega tipa (upoštevamo krajišča):
		$$\int_a^b f(x)\,dx ~=~ \sum_{i=0}^n A_i f(x_i) + \rr f$$
		\item Pravila zaprtega tipa (ne upoštevamo krajišč):
		$$\int_a^b f(x)\,dx ~=~ \sum_{i=1}^{n-1} A_i f(x_i) + \rr f$$
	\end{itemize}
	
	\item \textit{Gaussova} pravila: \\
	vozle in uteži določimo tako, da je pravilo največjega možnega reda

\end{itemize}

\end{definicija}
\vspace{0.5cm}

\begin{pravilo}[Trapezno pravilo]

Naj velja $n = 1$, $a = x_0$ in $b = x_1 = x_0 + h$. Potem sledi
$$\int_{x_0}^{x_1} f(x)\,dx ~=~ \frac{h}{2}(f(x_0) + f(x_1)) - \frac{h^3}{2}f''(\xi), ~~~\xi \in [x_0, x_1].$$
Za \textit{trapezno pravilo} velja 
$$\rr 1 = 0, ~~~\rr (x-x_0) = 0 ~~~\text{ter}~~~ \rr (x-x_0)^2 = \frac{h^3}{6} \neq 0,$$
torej je pravilo reda $1$. 

\end{pravilo}
\vspace{0.5cm}

\begin{pravilo}[Simpsonovo pravilo]

Naj velja $n = 2$, $a = x_0$, $x_1 = x_0 + h$ ter $x_2 = x_0 + 2h = b$, torej $h = \cfrac{b-a}{2}$. Potem sledi
$$\int_{x_0}^{x_2} f(x)\,dx ~=~ \frac{h}{3} (f(x_0) + 4f(x_1) + f(x_2)) - \frac{h^5}{90} f^{(4)}(\xi), ~~~\xi \in [x_0, x_2],$$ 
kjer je $f \in \C^4([x_0, x_2])$. \textit{Simpsonovo pravilo} je reda $3$.

\end{pravilo}
\vspace{0.5cm}

\begin{trditev}[Napaka aritmetike pri Newton-Cotesovih pravilih]

Recimo, da velja $|f(x_i) - \hat{f}(x_i)| < \varepsilon$. Ocena za napako aritmetike:
$$D_a ~=~ \left| \sum_{i=0}^n A_i f(x_i) - \sum_{i=0}^n A_i \hat{f}(x_i) \right| ~\leq~ \sum_{i=0}^n |A_i| |f(x_i) - \hat{f}(x_i)| ~\leq~ \varepsilon \sum_{i=0}^n |A_i|.$$
Ker so pravila točna za konstante velja
$$\int_a^b 1\,dx ~=~ \sum_{i=0}^n A_i \cdot 1 ~\Rightarrow~ \sum_{i=0}^n A_i ~=~ b-a.$$
Če so vse uteži pozitivne, potem $D_a \leq \varepsilon(b-a)$ in ni numeričnih težav.

\end{trditev}
\vspace{0.5cm}

% *************************************************************************************************

\subsection{Sestavljena integracijska pravila}
\vspace{0.5cm}

\begin{ideja}

Interval $[a, b]$ razdelimo na manjše podintervale in na vsakem podintervalu uporabimo integracijsko pravilo nizkega reda in rezultate seštejemo. 

\end{ideja}
\vspace{0.5cm}

\begin{pravilo}[Sestavljeno trapezno pravilo]

$h = \frac{b-a}{m}$
$$\int_a^b f(x)\,dx ~=~ \frac{h}{2} \left( f(x_0) + 2 \sum_{i=1}^{m-1} f(x_i) + f(x_m) \right) - \frac{h^2}{12} (b-a) f''(\mu)$$

\end{pravilo}
\vspace{0.5cm}

\begin{pravilo}[Sestavljeno Simpsonovo pravilo]

$n=2$, $h = \frac{b-a}{2m}$, $x_i = a + ih$, $i = 0, 1, \ldots, 2m$
$$\int_a^b f(x)\,dx ~=~ \frac{h}{3} \left( f(x_o) + 4 \sum_{i=0}^{m-1} f(x_{2i+1}) + 2 \sum_{i=0}^{m-1} f(x_{2i}) + f(x_{2m}) \right) - \frac{h^4}{180} (b-a) f^{(4)}(\mu)$$

\end{pravilo}
\vspace{0.5cm}

% *************************************************************************************************

\subsection{Ocena napake in Richardsonova ekstrapolacija}
\vspace{0.5cm}

\begin{metoda}

Označimo z $F_n$ približek za $I$, ki ga dobimo, o računamo s korakom $h$. Predpostavimo, da velja
$$I ~=~ F_n + c_0 \cdot h^p + \OO(h^{p+1}),$$
kjer je $c_0$ konstanta neodvisna od $h$. Z nekaj vragolijami dobimo
$$I ~=~ F_{\frac{h}{2}} + \underbrace{\frac{F_{\frac{h}{2}} - F_h}{2^p - 1}}_{\text{Ocena za napako približka $F_{\frac{h}{2}}$}} + \OO(h^{p+1}) ~=~ F_{h} + 2^p \underbrace{\frac{F_{\frac{h}{2}} - F_h}{2^p - 1}}_{\text{Ocena za napako približka $F_h$}} + \OO(h^{p+1})$$

\end{metoda}
\vspace{0.5cm}

% *************************************************************************************************

\subsection{Gaussova pravila}
\vspace{0.5cm}

\begin{ideja}

Vozle in uteži v integracijskemu pravilo izračunamo tako, da bo pravilo čim višjega reda.
$$\int_a^b f(x)\,dx ~=~ \sum_{i=0}^n A_i f(x_i) + \rr f, ~~~\text{neznanke}:~ A_i ~\text{in vozli}~ x_i$$
Če vozle izberemo tako, da bo $\omega \perp \p_n$, potem je red pravila enak $2n+1$. Velja tudi obratno.

\end{ideja}
\vspace{0.5cm}

\begin{opomba}[Drug način]

Iz baze $\{ 1, x, x^2, \ldots, x^n, x^{n+1} \}$ izračunamo ON bazo polinomov $p_0, p_1, \ldots, p_n, p_{n+1}$. Iz pogoja $\omega \perp \p_n$ vidimo, da za vozle $x_0, x_1, \ldots, x_n$ izberemo ničle polinoma $p_{n+1}$.

\end{opomba}
\vspace{0.5cm}

\begin{izrek}

Naj bo $\mathlarger{\F f = \sum_{i=0}^n A_i f(x_i)}$ Gaussovo integracijsko pravilo reda $2n+1$. Uteži
$$A_i ~=~ \int_a^b \ell_{i, n}(x)\,dx$$
so pozitivne. Za $f \in \C^{2n+2}([a, b])$ je napaka oblike
$$\rr f ~=~ \frac{f^{(2n+2)}(\xi)}{(2n+2)!} \int_a^b \omega^2(x)\,dx, ~~~\xi \in [a, b].$$
V integral, ki ga izračunamo, lahko vstavimo še pozitivno utež $\rho(x) > 0$.

\end{izrek}
\vspace{0.5cm}

% *************************************************************************************************

\pagebreak

% #################################################################################################

\section{Numerično reševanje NDE}
\vspace{0.5cm}

\begin{problem}[Začetni problem]
~\\
Začetni problem reda $1$:
\begin{align*}
y' ~&=~ f(x, y), ~~~f: [a, b] \times \R \rightarrow \R \\
y(a) ~&=~ y_a
\end{align*}
Začetni problem reda $p$:
\begin{align*}
y^{(p)} ~&=~ f(x, y, y', \ldots, y^{(p-1)}) \\
y(a) ~&=~ y_a, ~y'(a) ~=~ y_{a, 1}, ~y^2(a) ~=~ y_{a, 2}, ~\ldots, ~y^{(p-1)}(a) ~=~ y_{a, p-1}
\end{align*}

\end{problem}
\vspace{0.5cm}

\begin{problem}[Robni primer]

\begin{align*}
y^{(4)} &+ xy ~=~ 0 \\
y(a) ~&=~ y_a, ~y'(a) = y_{a, 1} \\
y(b) ~&=~ y_b, ~y'(b) = y_{b, 1}
\end{align*}

\end{problem}
\vspace{0.5cm}

\begin{metoda}[Eksplicitna Eulerjeva metoda]

$$y_{n+1} ~=~ y_n + h f(x_n, y_n), ~~~x_{n+1} = x_n + h$$

\end{metoda}
\vspace{0.5cm}

\begin{metoda}[Implicitna Eulerjeva metoda]

$$y_{n+1} ~=~ y_n + h f(x_{n+1}, y_{n+1})$$

\begin{lstlisting}
y$_{\texttt{n+1}}$ = g(y$_{\texttt{n+1}}$)     (g(y$_{\texttt{n+1}}$) = y$_{\texttt{n}}$ + h$\cdot$f(x$_{\texttt{n+1}}$, y$_{\texttt{n+1}}$))
y$_{\texttt{n+1}}^{\texttt{(0)}}$ = y$_{\texttt{n}}$
while |y$_{\texttt{n+1}}^{\texttt{(k)}}$ - y$_{\texttt{n+1}}^{\texttt{(k-1)}}$| $\geq$ tol$\cdot$|y$_{\texttt{n+1}}^{\texttt{(k)}}$|
    y$_{\texttt{n+1}}^{\texttt{(k)}}$ = g(y$_{\texttt{n+1}}^{\texttt{(k)}}$),     k = 1,2,...
\end{lstlisting}

\end{metoda}
\vspace{0.5cm}

\begin{metoda}[Trapezna metoda]

$x_n = a + n h$
$$y_{n+1} ~=~ y_n + \frac{h}{2} (f(x_n, y_n) + f(x_{n+1}, y_{n+1}))$$

\end{metoda}
\vspace{0.5cm}

% *************************************************************************************************

\subsection{Globalna in lokalna napaka}
\vspace{0.5cm}

\begin{definicija}

Metoda je \textit{konvergentna}, če za vse DE, ki zadoščajo pogojem eksistenčnega izreka, velja:
$$\|y - y_n\|_{\infty, [a, b]} \xrightarrow{h \rightarrow 0} 0,$$
oziroma:
$$\max_{0 \leq n \leq m} |y(x_n) - y_n| \xrightarrow{h \rightarrow 0 ~\text{oz.}~ m \rightarrow \infty} 0.$$

\end{definicija}
\vspace{0.5cm}

\begin{definicija}[Globalna napaka]

$$\max_{0 \leq n \leq m} |y(x_n) - y_n|$$
Globalna točka v točki $x_n$:
$$|y(x_n) - y_n|$$

\end{definicija}
\vspace{0.5cm}

\begin{definicija}

Metoda je reda $r$, če velja
$$\max_{0 \leq n \leq m} |y(x_n) - y_n| ~=~ C \cdot h^r + \OO(h^{r+1}) ~=~ \OO(h^r).$$

\end{definicija}
\vspace{0.5cm}

\begin{definicija}

\textit{Lokalna napaka} v točki $x_n$ je razlika med točno rešitvijo in njenim numeričnim približkom, ob predpostavki, da se ti dve rešitvi ujemata na vseh prejšnih korakih.
$$\T_n(n) ~=~ y(x_n) - y_n, ~~~\text{če}~ y(x_k) = y_k ~\forall k < n.$$

\end{definicija}
\vspace{0.5cm}

% *************************************************************************************************

\subsection{Runge-Kutta metode}
\vspace{0.5cm}

\begin{metoda}[$s$-stopenjska Runge-Kutta metoda]

Izračunamo $s$ koeficientov:
\begin{align*}
k_i ~&=~ h \cdot f(x_i + \alpha_i \cdot h, ~y_n = \sum_{j=1}^s \beta_{ij} k_j), ~~~i = 1, 2, \ldots, s \\
y_{n+1} ~&=~ y_n + \sum_{i=1}^s \gamma_i \cdot k_i
\end{align*}
Pri tem so $\alpha_i, \beta_{ij}, \gamma_i$ koeficienti, ki jih določimo tako, da je metoda čim višjega reda.
\begin{align*}
\alpha_i ~&\in~ [0, 1] \\
\alpha_i ~&=~ \sum_{j=1}^s \beta_{ij}
\end{align*}
Da bo R-K metoda reda vsaj $1$, mora biti $\mathlarger{\sum_{i=1}^s \gamma_i = 1}$.
R-K metode podamo v \textit{Butcherjevi shemi}: \\
\begin{center}
\begin{tabular}{c|cccc}
$\alpha_1$ & $\beta_{11}$ & $\beta_{12}$ & $\ldots$ & $\beta_{1s}$ \\
$\alpha_2$ & $\beta_{21}$ & $\beta_{22}$ & $\ldots$ & $\beta_{2s}$ \\
$\vdots$ & $\vdots$ & $\vdots$ & $\vdots$ & $\vdots$ \\ 
$\alpha_s$ & $\beta_{s1}$ & $\beta_{s2}$ & $\ldots$ & $\beta_{ss}$ \\ \hline
~ & $\gamma_1$ & $\gamma_2$ & $\ldots$ & $\gamma_s$
\end{tabular}
\end{center}
Če so $\beta_{ij} = 0$ $\forall j \geq i$, je metoda \textit{eksplicitna}. Če so $\beta_{ij} = 0$ $\forall j > i$ in je vsaj en $\beta_{ii} \neq 0$, je metoda \textit{diagonalno implicitna}. Sicer je \textit{implicitna}.

\end{metoda}
\vspace{0.5cm}

\begin{metoda}[Modificirana Eulerjeva metoda]
~\\
\begin{tabular}{c|cc}
$0$ & $0$ & $0$ \\
$\frac{1}{2}$ & $\frac{1}{2}$ & $0$ \\[1ex] \hline
$~$ & $0$ & $1$ 
\end{tabular} \\
Lokalna napaka je reda $3$, oz. metoda je reda $2$.

\end{metoda}
\vspace{0.5cm}
\begin{metoda}[Heunova metoda]
~\\
\begin{tabular}{c|cc}
$0$ & $0$ & $0$ \\
$1$ & $1$ & $0$ \\[1ex] \hline
$~$ & $\frac{1}{2}$ & $\frac{1}{2}$ 
\end{tabular} \\
Lokalna napaka je reda $3$.

\end{metoda}
\vspace{0.5cm}

\begin{metoda}
~\\
\begin{tabular}{c|cc}
$0$ & $0$ & $0$ \\
$1$ & $\frac{1}{2}$ & $\frac{1}{2}$ \\[1ex] \hline
$~$ & $\frac{1}{2}$ & $\frac{1}{2}$ 
\end{tabular} \\
Lokalna napaka je reda $3$. Metoda je \textit{diagonalno implicitna}.

\end{metoda}
\vspace{0.5cm}

\begin{metoda}[$4$-stopenjska R-K metoda]
~\\
\begin{tabular}{c|cccc}
$0$ & $0$ & $~$ & $~$ & $~$ \\
$\frac{1}{2}$ & $\frac{1}{2}$ & $0$ & $~$ & $~$ \\[1ex]
$\frac{1}{2}$ & $0$ & $\frac{1}{2}$ & $0$ & $~$ \\[1ex] 
$1$ & $0$ & $0$ & $1$ & $0$ \\ \hline
~ & $\frac{1}{6}$ & $\frac{2}{6}$ & $\frac{2}{6}$ & $\frac{1}{6}$
\end{tabular}
Lokalna napaka je reda $5$. Metoda je reda $4$.

\end{metoda}
\vspace{0.5cm}

% *************************************************************************************************

\pagebreak

% #################################################################################################

\end{document}