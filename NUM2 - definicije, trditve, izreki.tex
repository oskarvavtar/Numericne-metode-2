\documentclass[11pt]{article}
\usepackage[utf8]{inputenc}
\usepackage[slovene]{babel}

\usepackage{amsthm}
\usepackage{amsmath, amssymb, amsfonts}
\usepackage{relsize}
\usepackage{listings}
\lstset{
	basicstyle=\ttfamily,
	mathescape
}

\DeclareMathOperator{\dist}{dist}
\newcommand{\R}{\mathbb{R}}
\newcommand{\N}{\mathbb{N}}
\newcommand{\p}{\mathbb{P}}
\newcommand{\A}{\mathcal{A}}
\newcommand{\B}{\mathcal{B}}
\newcommand{\C}{\mathcal{C}}

\theoremstyle{definition}
\newtheorem{definicija}{Definicija}[section]

\theoremstyle{definition}
\newtheorem{problem}{Problem}[section]

\newtheorem{lema}{Lema}[section]
\newtheorem{trditev}{Trditev}[section]
\newtheorem{izrek}{Izrek}
\newtheorem*{dokaz}{Dokaz}
\newtheorem*{algoritem}{Algoritem}
\newtheorem*{posledica}{Posledica}
\newtheorem*{opomba}{Opomba}
\newtheorem*{metoda}{Metoda}

\title{Numerične metode 2 - definicije, trditve in izreki}
\author{Oskar Vavtar \\
po predavanjih profesorice Marjetke Knez}
\date{2020/21}

\begin{document}
\maketitle
\pagebreak
\tableofcontents
\pagebreak

% #################################################################################################

\section{Teorija aproksimacije}
\vspace{0.5cm}

\begin{definicija}[Aproksimacijska shema]

Z $X$ označimo (realni) vektorski prostor, katerega elemente želimo aproksimirati, $S \subseteq X$ označuje podprostor oz. podmnožico, v kateri iščemo aproksimant. \textit{Aproksimacijska shema} je operator $\A: X \rightarrow S$, ki vsakemu elementu $f \in X$ priredi aproksimacijski element $\tilde{f} = \A f \in S$.

\end{definicija}
\vspace{0.5cm}

\begin{definicija}[Optimalni aproksimacijski problem]

Naj bo $X$ vektorski prostor z normo $\|\bullet\|$, $S \subseteq X$. Za $f \in X$ iščemo $\tilde{f} \in S$, da je 
$$\|f - \tilde{f}\| ~=~ \inf_{s \in S}{\|f - s\|} ~=:~ \dist(f, S).$$

\end{definicija}
\vspace{0.5cm}

\begin{definicija}

Recimo, da je $S = S_n$, kjer je $n$ dimenzija. Zanima nas, ali za $f \in X$ in $\tilde{f_n} \in S_n$ napaka $\|f - \tilde{f_n}\|$ konvergira proti $0$, ko gre $n \rightarrow \infty$. Če je to res, je aproksimacijska shema \textit{konvergentna}.

Če gledamo zaporedje podprostorov $S_n \subset X$, mora veljati, da z večanjem svobodnih parametrov postane $S_n$ gost v $X$. Za polinome to sledi iz Weierstrassovega izreka.

\end{definicija}
\vspace{0.5cm}

\begin{izrek}[Weierstrassov izrek]

Naj bo $f \in \C([a, b])$ poljubna funkcija. Potem $\forall \varepsilon > 0$ obstaja polinom $p$, da je 
$$\|f - p\|_{\infty,~[a, b]} ~<~ \varepsilon.$$
Drugače povedano:
$$\dist_{\infty}(f, \p_n) ~\xrightarrow{n \rightarrow \infty}~ 0.$$

\end{izrek}
\vspace{0.5cm}

\begin{definicija}[Bernsteinov polinom]

$$\B_n f(x) ~=~ \sum_{i=0}^n f(\frac{i}{n}) \cdot B_i^n(x),$$
kjer je $B_i^n$ \textit{Bernsteinov bazni polinom}:
$$B_i^n(x) ~:=~ \binom{n}{i} x^i (1-x)^{n-1}, ~~~i = 0, 1, \ldots, n$$
Da se pokazati, da gre $\|f - \B_f\|_{\infty,~[0, 1]} \xrightarrow{n \rightarrow \infty} 0$. Bernsteinov aproksimacijski polinom nam poda en možen način aproksimacije funkcije $f$ (na $[0,1]$).

\end{definicija}
\vspace{0.5cm}

\begin{definicija}[Bernsteinov aproksimacijski operator]
~\\
$\B_n: \C([a, b]) \rightarrow \p_n$, ~$f \mapsto \B_n f$:
$$\B_n f(x) ~=~ \sum_{i=0}^n f \left( a + \frac{i}{n} (b-a) \right) \cdot B_i^n(\frac{x-a}{b-a})$$
$$\|f - \B_n f\|_{\infty, ~[a, b]} ~=~ \max_{x \in [a, b]} |f(x) - \B_n f(x)|$$ 

\end{definicija}
\vspace{0.5cm}

\pagebreak

% #################################################################################################

\section{Enakomerna aproksimacija \\zveznih funkcij s polinomi}
\vspace{0.5cm}

\begin{problem}

Za dano $f \in \C([a, b])$ iščemo polinom $p^* \in \p_n$, za katerega velja
$$\|f - p^*\|_{\infty,~[a, b]} ~=~ \min_{p \in \p_n} \|f - p\|_{\infty,~[a, b]} ~=~ \min_{p \in \p_n} \max_{x \in [a, b]} |f(x) - p(x)|.$$
Ta problem sodi pod optimalne aproksimacijske probleme. Polinom $p^*$ imenujemo \textit{polinom najboljše enakomerne aproksimacije}\footnote{Od tu naprej \textit{p.n.e.a.}} za $f$ na $[a, b]$. Problem je nelinearen.

\end{problem}
\vspace{0.5cm}

\begin{izrek}

Naj bo $f \in \C([a, b])$. Če je polinom $p \in \p_n$ takšen, da \textit{residual} $r = f - p$ doseže svojo normo $\|r\|_{\infty,~[a, b]}$ alternirajoče v vsaj $n+2$ točkah $x_i \in [a, b]$, $a \leq x_0 < x_1 < \ldots < x_{n+1} \leq b$, potem je $p$ \textit{p.n.e.a.} za $f$ na $[a, b]$. \\

\textbf{Natančneje:} Če obstaja $n+2$ točk $x_i \in [a, b]$, da je $\|r\|_{\infty, [a, b]} = |r(x_i)|$ za $i = 0, 1, \ldots, n+1$, in $r(x_i) \cdot r(x_{i+1}) < 0$ za $i = 0, 1, \ldots, n$, potem je $p \in \p_n$ \textit{p.n.e.a.} za $f$ na $[a, b]$.

\end{izrek}
\vspace{0.5cm}

\begin{definicija}

Naj bo $E = \{x_i;~a \leq x_0 < x_1 < \ldots < x_{n+1} \leq b\}$. Definirajmo \textit{minimaks} za $f$ na $E$:
$$M_n(f, E) ~=~ \min_{p \in \p_n} \max_{x \in E} |f(x) - p(x)|$$
Polinom, pri katerem je ta minimum dosežen, imenujemo \textit{p.n.e.a.} za $f$ na množici $E$. Dobimo ga tako, da rešimo sistem linearnih enačb:
$$f(x_i) - p(x_i) ~=~ (-1)^i m, ~~~i = 0, 1, \ldots, n+1$$
kjer so neznanke koeficienti polinoma $p$ ter število $m$.\footnote{Imamo $n+2$ enačb za $n+2$ neznank.}

\pagebreak

\begin{algoritem}[Remesov postopek]

Vhodni podatki: funkcija $f$, interval $[a, b]$, stopnja $n$, toleranca $\varepsilon$ \\

Ponavljaj $k = 0, 1, 2, \ldots$
\begin{enumerate}
	\item Poišči polinom ${p_k}^* \in \p_n$, ki zadošča pogojem
	$$f(x_i) - {p_k}^*(x_i) ~=~ (-1)^i m, ~~~ i = 0, 1, \ldots, n+2$$
	
	\item Poišči ekstrem residuala $r_l = f - {p_k}^*$, torej poišči $u \in [a, b]$, da bo
	$$|r_k(u)| ~=~ \|r_k\|_{\infty,~[a, b]}$$
	
	\item Če je $|r_k(u)| - |m| < \varepsilon$, potem končaj in vrni $p^* = {p_k}^*$. 
\end{enumerate}
\end{algoritem}
\vspace{0.5cm}

\begin{opomba}

Da se dokazati, da zaporedje polinomov, ki ga tvori Remesov postopek konvergira proti \textit{p.n.e.a.} $p^*$. Hitrost konvergence je \textit{linearna}.

\end{opomba}
\vspace{0.5cm}

\end{definicija}
\vspace{0.5cm}

\pagebreak

% #################################################################################################

\section{Aproksimacija po metodi najmanjših kvadratov}
\vspace{0.5cm}

\begin{definicija}

Naj bo $X$ vektorski prostor nad $\R$ s skalarnim produktom $\langle \bullet, \bullet \rangle$, s kvadratno normo $\|\bullet\|_2 = \sqrt{\langle \bullet, \bullet \rangle}$. $S \subseteq X$ je končnodimenzionalen podprostor v $X$, definiran kot
$$S ~=~ \mathcal{L}\textit{in}\{\varphi_1, \varphi_2, \ldots, \varphi_n\}, ~~~\dim{S} = n.$$
Za izbran $f \in X$ iščemo $f* \in S$, da velja
$$\|f - f^*\|_2 ~=~ \min_{s \in S}\|f - s\|_2.$$
$f^*$ imenujemo \textit{element najboljše aproksimacije po metodi najmanjših kvadratov.}\footnote{\textit{e.n.a. po MNK}}

\end{definicija}
\vspace{0.5cm}

\begin{izrek}

Naj bo $S \subseteq X.$ Element $f^* \in S$ je element najboljše aproksimacije po MNK za $f \in X$ natanko tedaj, ko je $f - f^* \perp S$.

\end{izrek}
\vspace{0.5cm}

\begin{posledica}

Iz izreka sledi konstrukcija. Naj bodo $\varphi_1, \varphi_2, \ldots, \varphi_n$ baza podprostora $S$. 
$$f^* ~=~ \sum_{j=1}^n \alpha_j \varphi_j,$$ 
$(\alpha_j)_{j=1}^n$ so neznani koeficienti. Veljati mora $f - f^* \perp S$, torej $f - f^* \perp \varphi_i$ $\forall i = 1, 2, \ldots, n$. Na podlagi tega dobimo
$$\begin{bmatrix}
\langle \varphi_1, \varphi_2 \rangle & \cdots & \langle \varphi_n, \varphi_1 \rangle \\
\vdots & \ddots & \vdots \\
\langle \varphi_1, \varphi_n \rangle & \cdots & \langle \varphi_n, \varphi \rangle
\end{bmatrix} \begin{bmatrix}
\alpha_1 \\
\vdots \\
\alpha_n
\end{bmatrix} ~=~ \begin{bmatrix}
\langle f, \varphi_1 \rangle \\
\vdots \\
\langle f, \varphi_n \rangle
\end{bmatrix}.$$
Ta sistem enačb imenujemo \textit{Gramov} oz. \textit{normalni sistem}. Numerilno ta sistem rešimo z razcepom Choleskega. Levo matriko imenujemo \textit{Gramova matrika}. Gramova matrika $G = (\langle \varphi_j, \varphi_i \rangle)_{i,j=1}^n$ je \textit{simetrična pozitivno definitna} matrika.

\end{posledica}
\vspace{0.5cm}

\pagebreak

\begin{algoritem}[Gram-Schmidt]

Vhodni podatki: baza $\{\psi_1, \psi_2, \ldots, \psi_n\}$. \\
Izhod: Ortonormirana baza $\{\varphi_1, \varphi_2, \ldots, \varphi_n\}$.
\begin{lstlisting}
for i = 1:n
	$\varphi_\texttt{i}$ = $\psi_\texttt{i}$
end
for i = 1:n
	$\varphi_\textit{i}$ = $\frac{\varphi_\texttt{i}}{\|\varphi_\texttt{i}\|_{\texttt{2}}}$
	for j = i+1:n
		$\varphi_\texttt{j}$ = $\varphi_\texttt{j}$ - $\langle \varphi_\texttt{j}$, $\varphi_\texttt{i} \rangle \varphi_\texttt{i}$
	end
end
\end{lstlisting}

\end{algoritem}
\vspace{0.5cm}

\pagebreak

% #################################################################################################

\section{Interpolacija}
\vspace{0.5cm}

\begin{problem}

Podane so vrednosti izbrane funkcije $f$ v $n+1$ paroma različnih točkah $x_0, x_1, \ldots, x_n$ na realni osi\footnote{To so \textit{interpolacijske točke}.}, iščemo pa neko preprostejšo funkcijo $g$, ki zadošča pogojem
$$f(x_i) ~=~ g(x_i), ~~~i = 0, 1, \ldots, n.$$

\end{problem}
\vspace{0.5cm}

\begin{problem}[Polinomska interpolacija]

Imejmo funkcijo $f \in \C([a, b])$ in zaporedje točk \\$a \leq x_0 < x_1 < \ldots < x_n \leq b$. Iščemo polinom $p = a_0 + a_1 x + \ldots + a_n x^n \in \p_n$, ki zadošča pogojem
$$p(x_i) ~=~ f(x_i), ~~~i = 0, 1, \ldots, n.$$

\end{problem}
\vspace{0.5cm}

% *************************************************************************************************

\subsection{Lagrangeeva oblika zapisa interpolacijskega polinoma}
\vspace{0.5cm}

\begin{definicija}[Lagrangeevi bazni polinomi]

\begin{align*}
\ell_{0, n}(x) ~&=~ \frac{(x-x_1)(x-x_2)\ldots(x-x_n)}{(x_0-x_1)(x_0-x_2)\ldots(x_0-x_n)}\\
\ell_{1, n}(x) ~&=~ \frac{(x-x_0)(x-x_2)\ldots(x-x_n)}{(x_1-x_0)(x_1-x_2)\ldots(x_1-x_n)}\\
&~~\vdots \\
\ell_{n, n}(x) ~&=~ \frac{(x-x_0)(x-x_1)\ldots(x-x_{n-1})}{(x_n-x_0)(x_n-x_1)\ldots(x_n-x_{n-1})}
\end{align*}
$i$-ti Lagrangeev bazni polinom lahko posplošimo kot
$$\ell_{i, n}(x) ~=~ \prod_{\substack{j=0 \\ j \neq i}}^n \frac{x-x_j}{x_i-x_j}, ~~~i = 0, 1, \ldots, n$$
Velja:
$$\ell_{i, n}(x_j) ~=~ \begin{cases}
1; ~&i=j \\
0; ~&i \neq j
\end{cases}$$

\end{definicija}
\vspace{0.5cm}

\begin{lema}

Polinomi $\ell_{i, n}$, $i = 0, 1, \ldots, n$ so baza za $\p_n$.

\end{lema}
\vspace{0.5cm}

\begin{trditev}[Lagrangeeva oblika zapisa interpolacijskega polinoma]

$$p(x) ~=~ \sum_{i=0}^n f(x_i) \ell_{i, n}(x)$$

\end{trditev}
\vspace{0.5cm}

\begin{lema}

Če je $f \in \p_n$, potem je
$$\sum_{i=0}^n f(x_i) \ell_{i, n}(x) ~=~ f(x).$$

\end{lema}
\vspace{0.5cm}

\begin{posledica}

Lagrangeevi bazni polinomi tvorijo \textit{particija} oz. \textit{razčlenitev} enote:
$$\sum_{i=0}^n \ell_{i,n}(x) ~=~ 1.$$

\end{posledica}
\vspace{0.5cm}

\begin{izrek}

Naj bo $a \leq x_0 < x_1 < \ldots < x_n \leq b$, $f \in \C^{n+1}([a, b])$ in \\$p(x) = \mathlarger{\sum_{i=0}^n f(x_i) \ell_{i, n}(x)}$ interpolacijski polinom za $f$ na točkah $x_0, x_1, \ldots, x_n$. Potem $\forall x \in [a, b]$ obstaka nek $\xi_x \in (a, b)$, da velja
\begin{align*}
f(x) - p(x) ~&=~ \frac{f^{(n+1)}(\xi_x)}{(n+1)!} \omega(x), \\
\omega(x) ~&=~ (x-x_0)(x-x_1)\ldots(x-x_n), ~~~\omega \in \p_{n+1}.
\end{align*}

\end{izrek}
\vspace{0.5cm}

\begin{opomba}

Za poljuben $x \in  [a, b]$ torej velja
$$\|f(x) - p(x)\|_{\infty,~[a, b]} ~\leq~ \frac{1}{(n+1)!} \|\omega(x)\|_{\infty,~[a, b]} \cdot \|f^{(n+1)}(x)\|_{\infty,~[a, b]}$$

\end{opomba}
\vspace{0.5cm}

% *************************************************************************************************

\subsection{Newtonova oblika zapisa interpolacijskega polinoma}
\vspace{0.5cm}

\begin{problem}

Za bazo v kateri bomo interpolacijski polinom izrazili, izberemo \textit{prestavljene potence}:
$$1, ~x-x_0, ~(x-x_0)(x-x_1), ~(x-x_0)(x-x_1)(x-x_2), ~\ldots, ~(x-x_0)(x-x_1)\ldots(x-x_{n-1})$$
$$p(x) ~=~ \sum_{i=0}^n c_i (x-x_0)(x-x_1)\ldots(x-x_{i-1})$$
Iščemo $c_i$, $i = 0, 1, \ldots, n$, da bo $p(x_i) = f(x_i)$ $\forall i$.

\end{problem}
\vspace{0.5cm}

\begin{definicija}

\textit{Deljiva diferenca} $[x_0, x_1, \ldots, x_k]f$ je \textit{vodilni koeficient} interpolacijskega polinoma stopnje $k$\footnote{Koeficient pri potenci $x^k$.}, ki se s funkcijo $f$ ujema v točkah $x_0, x_1, \ldots, x_k$. Sledi
$$p_l(x) ~=~ p_{k-1}(x) ~+~ [x_0, x_1, \ldots, x_k]f \,(x-x_0)(x-x_1)\ldots(x-x_{k-1}).$$

\end{definicija}
\vspace{0.5cm}

\begin{trditev}[Newtonova oblika zapisa interpolacijskega polinoma]

$$p(x) ~=~ \sum_{i=0}^n [x_0, x_1, \ldots, x_i]f \,(x-x_0)(x-x_1)\ldots(x-x_{i-1})$$

\end{trditev}
\vspace{0.5cm}

\begin{izrek}[Rekurzivna formula za deljene diference]

Naj bodo $x_0, x_1, \ldots, x_k$ paroma različne točke na $x$-osi. Tedaj je
$$[x_0, x_1, \ldots, x_k]f ~=~ \frac{[x_1, x_2, \ldots, x_k]f - [x_0, x_1, \ldots, x_{k-1}]f}{x_k - x_0}.$$

\end{izrek}
\vspace{0.5cm}

\pagebreak
\begin{algoritem}[Hornerjev algoritem]

Vhodni podatki: 
\begin{itemize}
	\item $x_0, x_1, \ldots, x_n$
	\item $d_0, x_1, \ldots, d_n$
	\item $x$
\end{itemize}
\begin{lstlisting}
v$_\texttt{n}$ = d$_\texttt{n}$
for i = n-1$\,$:$\,$-1$\,$:$\,$0
    v$_\texttt{i}$ = d$_\texttt{i}$ + (x-x$_\texttt{i}$)$\cdot$v$_{i+1}$
end
\end{lstlisting}
Izhod: $v_0$

\end{algoritem}
\vspace{0.5cm}

% *************************************************************************************************

\pagebreak

% #################################################################################################

\end{document}