\documentclass[11pt]{article}
\usepackage[utf8]{inputenc}
\usepackage[slovene]{babel}

\usepackage{amsthm}
\usepackage{amsmath, amssymb, amsfonts}
\usepackage{relsize}
\usepackage{listings}
\lstset{
	basicstyle=\ttfamily,
	mathescape
}

\DeclareMathOperator{\dist}{dist}
\newcommand{\R}{\mathbb{R}}
\newcommand{\N}{\mathbb{N}}
\newcommand{\p}{\mathbb{P}}
\newcommand{\A}{\mathcal{A}}
\newcommand{\B}{\mathcal{B}}
\newcommand{\C}{\mathcal{C}}

\theoremstyle{definition}
\newtheorem{definicija}{Definicija}[section]

\theoremstyle{definition}
\newtheorem{problem}{Problem}[section]

\newtheorem{lema}{Lema}
\newtheorem{trditev}{Trditev}
\newtheorem{izrek}{Izrek}
\newtheorem*{dokaz}{Dokaz}
\newtheorem*{algoritem}{Algoritem}
\newtheorem*{posledica}{Posledica}
\newtheorem*{opomba}{Opomba}
\newtheorem*{metoda}{Metoda}

\title{Numerične metode 2 - definicije, trditve in izreki}
\author{Oskar Vavtar \\
po predavanjih profesorice Marjetke Knez}
\date{2020/21}

\begin{document}
\maketitle
\pagebreak
\tableofcontents
\pagebreak

% #################################################################################################

\section{TEORIJA APROKSIMACIJE}
\vspace{0.5cm}

\begin{definicija}[Aproksimacijska shema]

Z $X$ označimo (realni) vektorski prostor, katerega elemente želimo aproksimirati, $S \subseteq X$ označuje podprostor oz. podmnožico, v kateri iščemo aproksimant. \textit{Aproksimacijska shema} je operator $\A: X \rightarrow S$, ki vsakemu elementu $f \in X$ priredi aproksimacijski element $\tilde{f} = \A f \in S$.

\end{definicija}
\vspace{0.5cm}

\begin{definicija}[Optimalni aproksimacijski problem]

Naj bo $X$ vektorski prostor z normo $\|.\|$, $S \subseteq X$. Za $f \in X$ iščemo $\tilde{f} \in S$, da je 
$$\|f - \tilde{f}\| ~=~ \inf_{s \in S}{\|f - s\|} ~=:~ \dist(f, S).$$

\end{definicija}
\vspace{0.5cm}

\begin{definicija}

Recimo, da je $S = S_n$, kjer je $n$ dimenzija. Zanima nas, ali za $f \in X$ in $\tilde{f_n} \in S_n$ napaka $\|f - \tilde{f_n}\|$ konvergira proti $0$, ko gre $n \rightarrow \infty$. Če je to res, je aproksimacijska shema \textit{konvergentna}.

Če gledamo zaporedje podprostorov $S_n \subset X$, mora veljati, da z večanjem svobodnih parametrov postane $S_n$ gost v $X$. Za polinome to sledi iz Weierstrassovega izreka.

\end{definicija}
\vspace{0.5cm}

\begin{izrek}[Weierstrassov izrek]

Naj bo $f \in \C([a, b])$ poljubna funkcija. Potem $\forall \varepsilon > 0$ obstaja polinom $p$, da je 
$$\|f - p\|_{\infty,~[a, b]} ~<~ \varepsilon.$$
Drugače povedano:
$$\dist_{\infty}(f, \p_n) ~\xrightarrow{n \rightarrow \infty}~ 0.$$

\end{izrek}
\vspace{0.5cm}

\begin{definicija}[Bernsteinov polinom]

$$\B_n f(x) ~=~ \sum_{i=0}^n f(\frac{i}{n}) \cdot B_i^n(x),$$
kjer je $B_i^n$ \textit{Bernsteinov bazni polinom}:
$$B_i^n(x) ~:=~ \binom{n}{i} x^i (1-x)^{n-1}, ~~~i = 0, 1, \ldots, n$$
Da se pokazati, da gre $\|f - \B_f\|_{\infty,~[0, 1]} \xrightarrow{n \rightarrow \infty} 0$. Bernsteinov aproksimacijski polinom nam poda en možen način aproksimacije funkcije $f$ (na $[0,1]$).

\end{definicija}
\vspace{0.5cm}

\begin{definicija}[Bernsteinov aproksimacijski operator]
~\\
$\B_n: \C([a, b]) \rightarrow \p_n$, ~$f \mapsto \B_n f$:
$$\B_n f(x) ~=~ \sum_{i=0}^n f \left( a + \frac{i}{n} (b-a) \right) \cdot B_i^n(\frac{x-a}{b-a})$$
$$\|f - \B_n f\|_{\infty, ~[a, b]} ~=~ \max_{x \in [a, b]} |f(x) - \B_n f(x)|$$ 

\end{definicija}
\vspace{0.5cm}

% #################################################################################################

\end{document}